\section{Desafio}
    \label{sec:desa}
Ao utilizar pares de imagens correspondentes deve-se buscar o entendimento e a exploração sobre a visão estéreo ao: extrair mapas de disparidade e de profundidade; utilizar os dados de calibração das câmeras e medição de objetos em 3D. Para tanto, três requisitos devem ser atendidos:

\begin{enumerate}
    \item \textbf{Estimativa de mapa de profundidade a partir de imagens estéreo retificadas}: manipulando, pelo menos, os conjuntos de  imagens~\emph{``Jadeplant''} e~\emph{``Playtable''} da base de imagens de~\emph{Middleburry} de $2014$~\cite{Middleburry2014} da configuração perfeita para computar os mapas de disparidade e de profundidade; a métrica BAD2.0 deve ser empregada;
    \item \textbf{Câmeras estéreo com convergência}: manipulando, pelo menos, os conjuntos de imagens~\emph{``Morpheus''} não retificado da base de imagens produzida por Furukawa e Ponce (indisponível) para computar o mapa de disparidade;
    \item \textbf{Paralelepípedo}: manipulando os mapas obtidos com o requisito anterior para computar a menor caixa que comporta os cliques de mouse fornecidos.
\end{enumerate}