\section{Conclusão}
    \label{sec:conc}
Os algoritmos implementados se provaram desafiadores seja pela dificuldade da programação seja por fatores intrínsecos. Para o Requisito~\ref{subsec:result_req1}, o algoritmo~\emph{OpenCV} apresentou os melhores resultados visuais, por ter conseguido conectar os blocos relacionados, mas os valores de disparidade destoam do~\emph{ground truth}, o que justifica o seu baixo desempenho. Entretanto, ao normalizar os mapas de disparidade a taxa de acertos aumentou, justamente pela compactação das escalas distintas numa comum o que aumenta o número de boas ``colisões'' de disparidade, esta pode ser uma boa técnica a ser utilizada para quando os limites de disparidade são previamente conhecidos, como no caso em questão, pois poderiam ser escalonados à faixa conhecida.

Ao se deparar com o Requisito~\ref{subsec:result_req2}, entretanto, houve a constatação da fragilidade dos algoritmos desenvolvidos, sem considerar as influências do pré-processamento das imagens, devido a falta de solução para o problema apresentado.